\documentclass[a4paper,headings=small]{scrartcl}
\KOMAoptions{DIV=12}

\usepackage[utf8x]{inputenc}
\usepackage{amsmath}
\usepackage{graphicx}
\usepackage{multirow}
\usepackage{listings}
\usepackage{subfigure}

% define style of numbering
\numberwithin{equation}{section} % use separate numbering per section
\numberwithin{figure}{section}   % use separate numbering per section

% instead of using indents to denote a new paragraph, we add space before it
\setlength{\parindent}{0pt}
\setlength{\parskip}{10pt plus 1pt minus 1pt}

\title{Photogrammetric Computer Vision - WS12/13 \\ Excercise 5 \\ \emph{Orientation of an image pair}}
\author{Team A: Marcus Grum, Robin Vobruba, Marcus Pannwitz, Jens Jawer}
\date{\today}

\pdfinfo{%
  /Title    (Photogrammetric Computer Vision - WS12/13 - Excercise 5 - Orientation of an image pair)
  /Author   (Team A: Marcus Grum, Robin Vobruba, Marcus Pannwitz, Jens Jawer)
  /Creator  ()
  /Producer ()
  /Subject  ()
  /Keywords ()
}

% Simple picture reference
%   Usage: \image{#1}{#2}{#3}
%     #1: file-name of the image
%     #2: percentual width (decimal)
%     #3: caption/description
%
%   Example:
%     \image{myPicture}{0.8}{My huge house}
%     See fig. \ref{fig:myPicture}.
\newcommand{\image}[3]{
	\begin{figure}[htbp]
		\centering
		\includegraphics[width=#2\textwidth]{#1}
		\caption{#3}
		\label{fig:#1}
	\end{figure}
}
\newcommand{\generatedImgRootImg}{../resources/img}
\newcommand{\generatedImgRootTarget}{../../../target}


\begin{document}

\maketitle

In this exercise where an orientation of an image pair is realized using epipolar lines.
The epipolar geometry can be represented by the fundamental matrix
when we are using algebraic projective geometry. 
The task consisted of generating the fundamental matrix from selfmade images 
and finding with this the corresponding epipolar lines, 
such that a geometric space impression will become visible.

\section{Image Acquisition:}

For the beginning, we were asked to take pictures with a digital camera,
which show an appropriate observation object from two different perspectives.
We decided to take pictures from..., such that it can't change
during several fotos that were taken by the same camera.
Firstly, the object can be seen, ...
Secondly, ....
Those generated input pictures can be seen in fig. ~\ref{fig:Input pictures}.

\begin{figure}
  \subfigure[Left perspective.]{\frame{\includegraphics[width=0.49\textwidth]{\generatedImgRootImg/Ex5_2.png}}} \hfill
  \subfigure[Right perspective.]{\frame{\includegraphics[height=0.49\textwidth]{\generatedImgRootImg/Ex5_1.png}}}
  \caption{Input pictures using the same camera configuration.}
  \label{fig:Input pictures}
\end{figure}
  
In the fokus, one can see ...
We were asked to use a general convergent image arrangement...
 
\section{Homologous points:}

Then, we were asked to manually at least 8 homologous points x ↔ x' in the image pair.
Those point correspondences can be found in fig. ~\ref{fig:Point Correspondences}
considering each input image.

\begin{figure}
  \subfigure[Left perspective.]{
    \begin{tabular}{ l || c | c ||}
	&\multicolumn{2}{|c||}{Image points [pix]} \\
      No & x & y \\
      \hline
      1 &  748 &  248\\
      2 & 2370 &  346\\
      3 & 3982 &  270\\
      4 &  886 & 1964\\
      5 & 2354 & 1612\\
      6 & 3790 & 1991\\
      7 & 3784 & 2189\\
      8 & 3784 & 2189
    \end{tabular}
  } \hfill
  \subfigure[Right perspective.]{
    \begin{tabular}{ l || c | c ||}
	&\multicolumn{2}{|c||}{Image points [pix]} \\
      No & x' & y'\\
      \hline
      1 & 748  & 484 \\
      2 & 2136 & 297 \\
      3 & 3883 & 512 \\
      4 & 803  & 2167\\
      5 & 2299 & 1782\\
      6 & 3784 & 2189\\
      7 & 3784 & 2189\\
      8 & 3784 & 2189
    \end{tabular}
  }
  \subfigure[Left perspective.]{\frame{\includegraphics[width=0.49\textwidth]{\generatedImgRootTarget/PointCorrespondencesL.png}}} \hfill
  \subfigure[Right perspective.]{\frame{\includegraphics[height=0.49\textwidth]{\generatedImgRootTarget/PointCorrespondencesR.png}}}
  \caption{Point Correspondences using Homologous points.}
  \label{fig:Point Correspondences}
\end{figure}

For a better understanding, those point correspondences have been 
visualized in fig. ~\ref{fig:Point Correspondences}
considering each input image as well.

\section{Computation of the Fundamental Matrix:}

Then we were asked to implement a function in C++ for the linear computation of the fundamental
matrix F using the normalized 8-point-algorithm.
This matrix F can be used to determine the epipolar line of the left picture
   \begin{align}
       l = F^T \cdot x'
   \end{align}
and it can be used to determine the epipolar line of the right picture
   \begin{align}
       l' = F \cdot x
   \end{align}

  \subsection{Conditioning}

  In the beginning, the interesting point correspondence matrices are computed for conditioning.
  One for each specific lens.
  These matrices include two transformations: 
  Firstly, the translation of the centroid of all points to the origin.

  Secondly, this function realizes the the scaling of the mean distance to origin to one.

  \subsection{Create design matrix}

  The design matrix is built as follows: \\
  \begin{align}
  A_{i}=
  \left( \begin{array}{ccccccccc}
  \hat{x_1}\hat{x_1}' & \hat{y_1}\hat{x_1}' & \hat{x_1}' & \hat{x_1}\hat{y_1}' & \hat{y_1}\hat{y_1}' & \hat{y_1}' & \hat{x_1} & \hat{x_1}\hat{y_1} & 1\\
  \hat{x_2}\hat{x_2}' & \hat{y_2}\hat{x_2}' & \hat{x_2}' & \hat{x_2}\hat{y_2}' & \hat{y_2}\hat{y_2}' & \hat{y_2}' & \hat{x_2} & \hat{x_2}\hat{y_2} & 1\\
   : & : & : & : & : & : & : & : & : \\
  \hat{x_n}\hat{x_n}' & \hat{y_n}\hat{x_n}' & \hat{x_n}' & \hat{x_n}\hat{y_n}' & \hat{y_n}\hat{y_n}' & \hat{y_n}' & \hat{x_n} & \hat{x_n}\hat{y_n} & 1
  \end{array} \right) 
  \end{align}

  \subsection{Solve equation system with SVD and Reshape}

  The linear homogeneous equation system is solved with the Single Value Decomposition.
  \begin{align}
  Af=0, p=(a,b,c,d,e,f,g,h,j)^T 
  \end{align}

  Afterwards, the reshape is realized with this function as well.
  \begin{align}
  \tilde{F}=
  \left( \begin{array}{cccc}
  a & b & c \\
  d & e & f \\ 
  g & h & j
  \end{array} \right)
  \end{align}

  \subsection{Enforce singularity}
Hence $|\tilde{F}|\neq 0$, we try  to find an optimal version $\tilde{F}^*$, such that
  \begin{align}
  ||\tilde{F}-\tilde{F}^*=min||
  \end{align}
and
  \begin{align}
  |\tilde{F}^*=0|.
  \end{align}
We enforce the singularity in using a Singular Value Decomposition backwards.
In decompose the known $\tilde{F}$,
  \begin{align}
  \tilde{F}^*=\tilde{U} diag(d_1,d_2,d_3)\tilde{V}^T.
  \end{align}  
we set the last element of the diagonal matrix to 0, such that $\tilde{F}^*$ can be computed
as follows:
  \begin{align}
  \tilde{F}^*=\tilde{U} diag(d_1,d_2,0)\tilde{V}^T.
  \end{align}
  \subsection{Deconditioning}
In order to get back to the original coordinates, a final transformation matrix $H$ 
that is Euclidean normalized is calculated as follows:
\begin{align}
F=T'^{-1}\tilde{F}^{*}T
\end{align}

\section{Visualization of the fundamental matrix by epipolar lines:}

Point correspondences for both input pictures, 
as they have been used for the epipolar line search 
can be found in fig. ~\ref{fig:Point Correspondences}.

Their associated epipolar lines can be found in fig. ~\ref{fig:Epipolar Lines}.
\begin{figure}
  \subfigure[Left perspective.]{\frame{\includegraphics[width=0.49\textwidth]{\generatedImgRootTarget/EpipolarLinesL.png}}} \hfill
  \subfigure[Right perspective.]{\frame{\includegraphics[height=0.49\textwidth]{\generatedImgRootTarget/EpipolarLinesR.png}}}
  \caption{Epipolar lines using given point correspondences.}
  \label{fig:Epipolar Lines}
\end{figure}
\section{Evaluation:}
As we can see in the image pair of fig. ~\ref{fig:Point Correspondences},
the line characteristics are commented in brief in the following:
...

Based on the computed fundamental matrix $F$, the geometric image error
is calculated for all points. It is also called Sampson distance
and measures the symmetric epipolar distance.
\begin{align}
  d_{Sampson} = \frac{1}{N} \sum_{i=1}^N \frac{(x'_i F x_i)^2}{(F x_i)_a^2 + (F x_i)_b^2 + (F^T x'_i)_a^2 + (F^T x'_i)_b^2}
\end{align}
As $i$ is standing for each point, $N$ is the number of total points, 
which was in our case 8, $x$ and $x'$ are standing for the image coordinates of a certain
point and $a$ or $b$ are standing for first or second entry of corresponding vector.

The Sampson distance in our case was $d_{Sampson}=...$
\section{Printed Code:}

%\lstset{language=<C++>}
%\begin{lstlisting}
%test
%\end{lstlisting}
\lstinputlisting[breaklines=true]{../native/pcv5.cpp}
\end{document}

