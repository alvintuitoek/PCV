\documentclass[a4paper,headings=small]{scrartcl}
\KOMAoptions{DIV=12}

\usepackage[utf8x]{inputenc}
\usepackage{amsmath}
\usepackage{graphicx}
\usepackage{multirow}
\usepackage{listings}

% define style of numbering
\numberwithin{equation}{section} % use separate numbering per section
\numberwithin{figure}{section}   % use separate numbering per section

% instead of using indents to denote a new paragraph, we add space before it
\setlength{\parindent}{0pt}
\setlength{\parskip}{10pt plus 1pt minus 1pt}

\title{Photogrammetric Computer Vision - WS12/13 \\ Excercise 1 \\ \emph{Load, Modify and Save Images with OpenCV}}
\author{Team a: Marcus Grum, Robin Vobruba, Markus Pannwitz, Jens Jawer}
\date{\today}

\pdfinfo{%
  /Title    (Photogrammetric Computer Vision - WS12/13 - Excercise 1 - Load, Modify and Save Images with OpenCV)
  /Author   (Team a: Marcus Grum, Robin Vobruba, Markus Pannwitz, Jens Jawer)
  /Creator  ()
  /Producer ()
  /Subject  ()
  /Keywords ()
}

% Simple picture reference
%   Usage: \image{#1}{#2}{#3}
%     #1: file-name of the image
%     #2: percentual width (decimal)
%     #3: caption/description
%
%   Example:
%     \image{myPicture}{0.8}{My huge house}
%     See fig. \ref{fig:myPicture}.
\newcommand{\image}[3]{
	\begin{figure}[htbp]
		\centering
		\includegraphics[width=#2\textwidth]{#1}
		\caption{#3}
		\label{fig:#1}
	\end{figure}
}


\begin{document}


\maketitle



\section{Points and Lines in the Plane}


\subsection{}

You would like to compute the connecting line between two 2D points. \newline
What happens, if the two points are identical? \newline





\subsection{}

Where does the general line $ x cos \phi + y sin \phi = d$ intersect the line $(0, 0, 1)^T$
given in homogeneous coordinates? \newline
How can this point be interpreted? \newline


\subsection{}

Show that the horizon is a straight line by showing that three points
on the horizon are always collinear. \newline


\section{C++ and OpenCV}


\subsection{}

The two points $x = (2, 3)^T$ and $y = (-4, 5)^T$ are given. \newline
a) Determine the connecting line $l$ between the two points. \newline

      $l=(-2,-6, 22)^T$

b) Move $x$ and $y$ in the direction $t = (6, -7)^T$,
rotate afterwards using the angle $\phi = 15°$ and finally
scale with factor $\Lambda = 8$. \newline

\[T= \left( \begin{array}{ccc}
1 & 0 &  6 \\
0 & 1 & -7 \\
0 & 0 &  1  \end{array} \right)\] 

\[R= \left( \begin{array}{ccc}
  0.96592581 & -0.25881904 & 0 \\
  0.25881904 &  0.96592581 & 0 \\
  0          &  0          & 1  \end{array} \right)\] 

\[S= \left( \begin{array}{ccc}
8 & 0 & 0 \\
0 & 8 & 0 \\
0 & 0 & 1 \end{array} \right)\] 

$x'=x_{new}=(21.454813; 16.18222; 1)^T$ \newline
$y'=y_{new}=(-24.909626; 31.637032; 1)^T$ \newline

c) Accomplish the same operations with the line $l$. \newline

      $l'=l_{new}=(-0.25881904; -0.77645713; 18.117714)^T$




\subsection{}

Examine whether the transformed points $x’$ and $y’$ are
on the transformed line $l’$. \newline

Considering a tollerance eps of (1e-05), both points $x_{new}$ and $y_{new}$ 
are on the line $l_{new}$.

\end{document}
