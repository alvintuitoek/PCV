\documentclass[a4paper,headings=small]{scrartcl}
\KOMAoptions{DIV=12}

\usepackage[utf8x]{inputenc}
\usepackage{amsmath}
\usepackage{graphicx}
\usepackage{multirow}
\usepackage{listings}

% define style of numbering
\numberwithin{equation}{section} % use separate numbering per section
\numberwithin{figure}{section}   % use separate numbering per section

% instead of using indents to denote a new paragraph, we add space before it
\setlength{\parindent}{0pt}
\setlength{\parskip}{10pt plus 1pt minus 1pt}

\title{Photogrammetric Computer Vision - WS12/13 \\ Excercise 1 \\ \emph{Load, Modify and Save Images with OpenCV}}
\author{Team a: Marcus Grum, Robin Vobruba, Markus Pannwitz, Jens Jawer}
\date{\today}

\pdfinfo{%
  /Title    (Photogrammetric Computer Vision - WS12/13 - Excercise 1 - Load, Modify and Save Images with OpenCV)
  /Author   (Team a: Marcus Grum, Robin Vobruba, Markus Pannwitz, Jens Jawer)
  /Creator  ()
  /Producer ()
  /Subject  ()
  /Keywords ()
}

% Simple picture reference
%   Usage: \image{#1}{#2}{#3}
%     #1: file-name of the image
%     #2: percentual width (decimal)
%     #3: caption/description
%
%   Example:
%     \image{myPicture}{0.8}{My huge house}
%     See fig. \ref{fig:myPicture}.
\newcommand{\image}[3]{
	\begin{figure}[htbp]
		\centering
		\includegraphics[width=#2\textwidth]{#1}
		\caption{#3}
		\label{fig:#1}
	\end{figure}
}


\begin{document}


\maketitle



\section{Points and Lines in the Plane}


\subsection{}

You would like to compute the connecting line between two 2D points. \newline
What happens, if the two points are identical? \newline


$l=x \times y=
\left( \begin{array}{c}
x_2y_3-x_3y_2\\
x_3y_1-x_1y_3\\
x_1y_2-x_2y_1\end{array} \right)$
, where $y=x=(x_1,x_2,x_3)^T \newline
l=x \times y=
\left( \begin{array}{c}
x_2x_3-x_3x_2\\
x_3x_1-x_1x_3\\
x_1x_2-x_2x_1\end{array} \right)=(0,0, 0)^T=0$. \newline
This result of 0 can be used to flag that the points are identical, i.e., the result
is not a specific line but rather a set of lines. The resulting line is not unique: 
any line going through one point goes through the other as well.

When both points are at infinity, the resulting line would be a line at infinity called horizon.

%http://www.google.de/url?sa=t&rct=j&q=computer%20vision%20intersection%20of%20two%20identical%20points&source=web&cd=2&ved=0CCcQFjAB&url=http%3A%2F%2Fwww.cvl.isy.liu.se%2Feducation%2Fgraduate%2Fgeometry-for-computer-vision%2Flectures%2FLecture1.pdf&ei=n-CUUIi-E8jMsgaw0oCICQ&usg=AFQjCNEJOniNcH8rmH2m05dvyGma7YPjPg&cad=rja
%page 25

\subsection{}

Where does the general line $ x cos \phi + y sin \phi = d$ intersect the line $(0, 0, 1)^T$
given in homogeneous coordinates? \newline
How can this point be interpreted? \newline

The general line $l_1$ can be implicitly written like the following: $l_1=(cos \phi, sin \phi, -d)^T$. \newline
The second line $l_{2}=(0, 0, 1)^T$ can be seen as ideal line at infinity $l_{\infty}=(0, 0, c)^T$.

The intersection of the lines $l_1$ and $l_2$ in point $x_{1,2}$ can be found by calculating the cross product:

$x_{1,2}=l_1 \times l_2=
\left( \begin{array}{c}
x_2y_3-x_3y_2\\
x_3y_1-x_1y_3\\
x_1y_2-x_2y_1\end{array} \right)=
\left( \begin{array}{c}
sin \phi \cdot 1-(-d-0)\\
-d \cdot 0-cos \phi \cdot 1\\
cos \phi \cdot 0-sin \phi \cdot 0\end{array} \right)=
\left( \begin{array}{c}
sin \phi\\
-cos \phi\\
0\end{array} \right)$ = $x_{\infty}$ \newline

All general lines (but the horizon) intersect with the horizon at infinity 
at the vanishing point $x_{\infty}=(u,v,0)^T$. This result can be interpreted as follows:
Every parallel lines intersect at infinity at an proper vanishing point.
This fact was firstly used in works of realism art.

\subsection{}

Show that the horizon is a straight line by showing that three points
on the horizon are always collinear. \newline

The horizon line $l_1$ can be implicitly written like the following: $l_1=(0, 0, c)^T$. \newline
Three points on the horizon are the following ones: \newline
$x_{1}=(u-1,v-1,0)^T$, \newline
$x_{2}=(u,v,0)^T$, \newline
$x_{3}=(u+1,v+1,0)^T$. \newline

These points have to be collinear when lying on a straight line:

$det[x_{1},x_{2},x_{3}]=
det\left[ \begin{array}{ccc}
u-1 & u & u+1\\
v-1 & v & v+1\\
0 & 0 & 0\end{array} \right]=...\newline
...=(u-1) \cdot v \cdot0 - (u-1) \cdot (v+1) \cdot 0 + u \cdot (v+1) \cdot 0 - u \cdot (v-1) \cdot 0
+(u+1) \cdot (v-1) \cdot 0 - (u+1) \cdot v \cdot 0=... \newline
...=0.$ \newline

These three points are collinear because they are collinear. They are alle points of the horizon, 
hence, the horizon is a straight line.

\newpage
\section{C++ and OpenCV}


\subsection{}

The two points $x = (2, 3)^T$ and $y = (-4, 5)^T$ are given. \newline
a) Determine the connecting line $l$ between the two points. \newline

      $l=x \times y=(-2,-6, 22)^T$

b) Move $x$ and $y$ in the direction $t = (6, -7)^T$,
rotate afterwards using the angle $\phi = 15°$ and finally
scale with factor $\lambda = 8$. \newline

\[T= 
\left( \begin{array}{cc}
I & t  \\
0^T & 1   \end{array} \right)
=
\left( \begin{array}{ccc}
1 & 0 &  6 \\
0 & 1 & -7 \\
0 & 0 &  1  \end{array} \right)\] 

\[R= 
\left( \begin{array}{cc}
R & 0  \\
0^T & 1   \end{array} \right)
=
\left( \begin{array}{ccc}
  0.96592581 & -0.25881904 & 0 \\
  0.25881904 &  0.96592581 & 0 \\
  0          &  0          & 1  \end{array} \right)\] 

\[S= 
\left( \begin{array}{cc}
\lambda I & 0  \\
0^T & 1   \end{array} \right)
=
\left( \begin{array}{ccc}
8 & 0 & 0 \\
0 & 8 & 0 \\
0 & 0 & 1 \end{array} \right)\] 

$x'=x_{new}=S(R(Tx))=(21.454813; 16.18222; 1)^T$ \newline
$y'=y_{new}=S(R(Ty))=(-24.909626; 31.637032; 1)^T$ \newline

c) Accomplish the same operations with the line $l$. \newline

      $l'=l_{new}=S(R(Tl))=(-0.25881904; -0.77645713; 18.117714)^T$




\subsection{}

Examine whether the transformed points $x’$ and $y’$ are
on the transformed line $l’$. \newline

Considering a tollerance eps of (1e-05) because of the use of variables of the type float, both points $x_{new}$ and $y_{new}$ 
are on the line $l_{new}$, as can be seen in the following: \newline
$x_{new}$ is on line, hence $x_{new}^Tl_{new} \approx 0$ and $y_{new}$ is on line, hence $y_{new}^Tl_{new} \approx 0$. \newline

\end{document}
