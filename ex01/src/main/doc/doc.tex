\documentclass[a4paper,headings=small]{scrartcl}
\KOMAoptions{DIV=12}

\usepackage[utf8x]{inputenc}
\usepackage{amsmath}
\usepackage{graphicx}
\usepackage{multirow}
\usepackage{listings}

% define style of numbering
\numberwithin{equation}{section} % use separate numbering per section
\numberwithin{figure}{section}   % use separate numbering per section

% instead of using indents to denote a new paragraph, we add space before it
\setlength{\parindent}{0pt}
\setlength{\parskip}{10pt plus 1pt minus 1pt}

\title{Photogrammetric Computer Vision - WS12/13 \\ Excercise 1 \\ \emph{Load, Modify and Save Images with OpenCV}}
\author{Team A: Marcus Grum, Robin Vobruba, Markus Pannwitz, Jens Jawer}
\date{\today}

\pdfinfo{%
  /Title    (Photogrammetric Computer Vision - WS12/13 - Excercise 1 - Load, Modify and Save Images with OpenCV)
  /Author   (Team A: Marcus Grum, Robin Vobruba, Markus Pannwitz, Jens Jawer)
  /Creator  ()
  /Producer ()
  /Subject  ()
  /Keywords ()
}

% Simple picture reference
%   Usage: \image{#1}{#2}{#3}
%     #1: file-name of the image
%     #2: percentual width (decimal)
%     #3: caption/description
%
%   Example:
%     \image{myPicture}{0.8}{My huge house}
%     See fig. \ref{fig:myPicture}.
\newcommand{\image}[3]{
	\begin{figure}[htbp]
		\centering
		\includegraphics[width=#2\textwidth]{#1}
		\caption{#3}
		\label{fig:#1}
	\end{figure}
}


\begin{document}


\maketitle



\section{Points and Lines in the Plane}


\subsection{Question 1}


We would like to compute the connecting line between two 2D points. \\
What happens, if the two points are identical? \\

$l = x \times y =
\left( \begin{array}{c}
x_2 y_3 - x_3 y_2 \\
x_3 y_1 - x_1 y_3 \\
x_1 y_2 - x_2 y_1
\end{array} \right)$
, where $y = x = (x_1, x_2, x_3)^T \\
l = x \times y =
\left( \begin{array}{c}
x_2 x_3 - x_3 x_2\\
x_3 x_1 - x_1 x_3\\
x_1 x_2 - x_2 x_1\end{array} \right) = (0, 0, 0)^T$.

This result of $(0, 0, 0)^T$ can be used to defere that the points are identical.
Also, the result is not a specific line but rather a set of lines.
The resulting line is not unique:
any line going through one point, goes through the other point as well.


%http://www.google.de/url?sa=t&rct=j&q=computer%20vision%20intersection%20of%20two%20identical%20points&source=web&cd=2&ved=0CCcQFjAB&url=http%3A%2F%2Fwww.cvl.isy.liu.se%2Feducation%2Fgraduate%2Fgeometry-for-computer-vision%2Flectures%2FLecture1.pdf&ei=n-CUUIi-E8jMsgaw0oCICQ&usg=AFQjCNEJOniNcH8rmH2m05dvyGma7YPjPg&cad=rja
%page 25

\subsection{Question 2}

Where does the general line $ x cos \phi + y sin \phi = d$ intersect the line $(0, 0, 1)^T$
given in homogeneous coordinates?

How can this point be interpreted?

The general line $l_1$ can be implicitly written like the following:
$l_1=(cos \phi, sin \phi, -d)^T$.

The second line, $l_{2}=(0, 0, 1)^T$, can be seen as ideal line at infinity $l_{\infty}=(0, 0, c)^T$.

The intersection of the lines $l_1$ and $l_2$ in point $x_{1,2}$ can be found by calculating the cross product:

$x_{1,2} = l_1 \times l_2 =
\left( \begin{array}{c}
x_2 y_3 - x_3 y_2\\
x_3 y_1 - x_1 y_3\\
x_1 y_2 - x_2 y_1\end{array} \right) =
\left( \begin{array}{c}
sin \phi \cdot 1 - (-d - 0)\\
-d \cdot 0 - cos \phi \cdot 1\\
cos \phi \cdot 0 - sin \phi \cdot 0\end{array} \right) =
\left( \begin{array}{c}
sin \phi\\
-cos \phi\\
0\end{array} \right)$ = $x_{\infty}$

All general lines (except the horizon) intersect with the horizon at infinity,
at the vanishing point $x_{\infty}=(u,v,0)^T$.
This result can be interpreted as follows:
All parallel lines intersect at infinity at a proper vanishing point, 
but depend on the angle $\phi$.
This fact was firstly used in works of realism art.

\subsection{Question 3}

Show that the horizon is a straight line by showing that three points
on the horizon are always collinear.

The horizon line $l_1$ can be implicitly written like the following: $l_1=(0, 0, c)^T$.

Three points on the horizon are the following ones:

$x_{1} = (u - 1, v - 1, 0)^T$, \\
$x_{2} = (u, v, 0)^T$, \\
$x_{3} = (u + 1, v + 1, 0)^T$. \\

These points have to be collinear when lying on a straight line:

$det[x_{1}, x_{2}, x_{3}] =
det\left[ \begin{array}{ccc}
u-1 & u & u+1\\
v-1 & v & v+1\\
0 & 0 & 0\end{array} \right] = 0 \\
... = (u-1) \cdot v \cdot0 - (u-1) \cdot (v+1) \cdot 0 + u \cdot (v+1) \cdot 0 - u \cdot (v-1) \cdot 0
+(u+1) \cdot (v-1) \cdot 0 - (u+1) \cdot v \cdot 0=... \\
...= 0$.

These three points are collinear because their determinant equals zero.
They are all points of the horizon.
Hence, the horizon is a straight line.

\newpage
\section{C++ and OpenCV}


\subsection{}

The two points $x = (2, 3)^T$ and $y = (-4, 5)^T$ are given.

a) Determine the connecting line $l$ between the two points.

      $l=x \times y=(-2,-6, 22)^T$

b) Move $x$ and $y$ in the direction $t = (6, -7)^T$,
rotate afterwards using the angle $\phi = 15°$ and finally
scale with factor $\lambda = 8$.

\[T =
\left( \begin{array}{cc}
I & t \\
0^T & 1   \end{array} \right)
=
\left( \begin{array}{ccc}
1 & 0 &  6 \\
0 & 1 & -7 \\
0 & 0 &  1  \end{array} \right)\]

\[R =
\left( \begin{array}{cc}
R & 0  \\
0^T & 1   \end{array} \right)
=
\left( \begin{array}{ccc}
  0.96592581 & -0.25881904 & 0 \\
  0.25881904 &  0.96592581 & 0 \\
  0          &  0          & 1  \end{array} \right)\]

\[S =
\left( \begin{array}{cc}
\lambda I & 0  \\
0^T & 1   \end{array} \right)
=
\left( \begin{array}{ccc}
8 & 0 & 0 \\
0 & 8 & 0 \\
0 & 0 & 1 \end{array} \right)\]

$x'=x_{new} = S(R(Tx)) = (21.454813; 16.18222; 1)^T$ \\
$y'=y_{new} = S(R(Ty)) = (-24.909626; 31.637032; 1)^T$

c) Accomplish the same operations with the line $l$.

      $l' = l_{new} = S(R(Tl)) = (-0.25881904; -0.77645713; 18.117714)^T$




\subsection{}

Examine whether the transformed points $x’$ and $y’$ are
on the transformed line $l’$.

Considering a tollerance eps of (1e-05),
because of the use of variables of the type float.
Both points $x_{new}$ and $y_{new}$
are on the line $l_{new}$, as can be seen in the following:

$x_{new}$ is on line, hence $x_{new}^Tl_{new} \approx 0$ and $y_{new}$ is on line, hence $y_{new}^Tl_{new} \approx 0$.

\end{document}
