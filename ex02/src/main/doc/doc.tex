\documentclass[a4paper,headings=small]{scrartcl}
\KOMAoptions{DIV=12}

\usepackage[utf8x]{inputenc}
\usepackage{amsmath}
\usepackage{graphicx}
\usepackage{multirow}
\usepackage{listings}

% define style of numbering
\numberwithin{equation}{section} % use separate numbering per section
\numberwithin{figure}{section}   % use separate numbering per section

% instead of using indents to denote a new paragraph, we add space before it
\setlength{\parindent}{0pt}
\setlength{\parskip}{10pt plus 1pt minus 1pt}

\title{Photogrammetric Computer Vision - WS12/13 \\ Excercise 2 \\ \emph{Stich images together}}
\author{Team A: Marcus Grum, Robin Vobruba, Markus Pannwitz, Jens Jawer}
\date{\today}

\pdfinfo{%
  /Title    (Photogrammetric Computer Vision - WS12/13 - Excercise 2 - Stich images together)
  /Author   (Team A: Marcus Grum, Robin Vobruba, Markus Pannwitz, Jens Jawer)
  /Creator  ()
  /Producer ()
  /Subject  ()
  /Keywords ()
}

% Simple picture reference
%   Usage: \image{#1}{#2}{#3}
%     #1: file-name of the image
%     #2: percentual width (decimal)
%     #3: caption/description
%
%   Example:
%     \image{myPicture}{0.8}{My huge house}
%     See fig. \ref{fig:myPicture}.
\newcommand{\image}[3]{
	\begin{figure}[htbp]
		\centering
		\includegraphics[width=#2\textwidth]{#1}
		\caption{#3}
		\label{fig:#1}
	\end{figure}
}


\begin{document}

\maketitle

In this exercise a fundamental projective transformation is realized. 
The task consisted of rectifying images geometrically and stitching
them together, so that a panorama mosaic was generated.


\section{Image Acquisition:}

For the beginning, we were asked to take pictures with a digital camera,
which overlap horizontal at least 30 percent. 
We decided to use three images without disparities, 
as they can be seen in fig.\ref{fig:../../../target/InputPics/1},
fig.\ref{fig:../../../target/InputPics/2}, fig.\ref{fig:../../../target/InputPics/3}.

\image{../../../target/InputPics/1}{0.3}{%
		Input picture 1.}
\image{../../../target/InputPics/2}{0.3}{%
		Input picture 2.}
\image{../../../target/InputPics/3}{0.3}{%
		Input picture 3.}

Those pictures were taken by turning around the projection center of the camera
and focussing 5 ordinary breakfast items as they can be found in every house hold.
From the left to the right: a salt shaker, an egg, another egg, a bread and another bread.
The camera was positioned in the middle of a round table, such that it had a stable position
and could not disturb the algorithm. The pictures were clear and well exposed. 

\section{Correspondence Analysis:}

Those images were transfered into the computer. At least four corresponding image points 
x ↔ x′ between two neighbouring images were manually measured through clicking manually on the surface. 
Those points are described in the following.

The point pairs that are used for the adaption process of picture one and picture two
can be seen in fig. \ref{fig:../../../target/general_selection_1}.

\image{../../../target/general_selection_1}{0.6}{%
		Point selections of picture 1 and picture 2 (\emph{general positions}).}

Here, one can see that 6 independent point pairs are used.
For the second adaption process of picture three and picture two,
those point pairs are used that can be found in fig. \ref{fig:../../../target/general_selection_2}.

\image{../../../target/general_selection_2}{0.6}{%
		Point selections of picture 3 and picture 2 (\emph{general positions}).}

Here, one can see that 7 independent point pairs are used.
Although just 4 point pairs were required, we decided to choose more
because the most obvious points enter the food and its podest are nearly collinear
as the following fig. \ref{fig:../../../target/nonGeneral_selection_1} shall show.

\image{../../../target/nonGeneral_selection_1}{0.6}{%
		Point selections of picture 1 and picture 2 (\emph{with non general points}).}

The positions of the point pairs were not choosen randomly: They were chosen 
at remarkable obvious points, such as the edge between the food and its podest, 
between the podest and the table or the very top of the food.


\section{Homography Computation:}

In the following, a C++-function is implemented to estimate a 2D homography using singular value
decomposition. It computes the two necessary homography matrices.

  \subsection{Conditioning}

In the beginning, the two coordinate transformation matrices are computed for conditioning.
One for each base picture and one for each picture to be attached.
These matrices include two transformations: 
Firstly, the translation of the centroid of all points to the origin.
In the code, the following function realizes the transformations:
\begin{lstlisting}
Mat tBase = getCondition2D(base);
\end{lstlisting}
Secondly, this function realizes the the scaling of the mean distance to origin to one.

  \subsection{Create design matrix}

The design matrix is built as follows: \\
\begin{align}
A_{i}=
\left( \begin{array}{ccc}
-\tilde{w}'\tilde{x_i^T} & 0 & \tilde{u_i}'\tilde{x_i^T} \\
0 & -\tilde{w}'\tilde{x_i^T} & \tilde{v_i}'\tilde{x_i^T}
\end{array} \right) 
\end{align}

In the code, the following function realizes the design matrix:
\begin{lstlisting}
Mat getDesignMatrix_homography2D(Mat& base, Mat& attach);
\end{lstlisting}

  \subsection{Solve equation system with SVD and Reshape}

The linear homogeneous equation system is solved with the Single Value Decomposition.
\begin{align}
Ah=0, h=(a,b,c,d,e,f,g,h,j)^T 
\end{align}

In the code, the following function realizes the solution:
\begin{lstlisting}
Mat H = solve_dlt(design);
\end{lstlisting}

This function uses the function compute(A, d, u, vt, 0); that computes U, D and V matrices.
D and V are in decending order, such that the last row of V stands for the parameters of $H$. 

Afterwards, the reshape is realized with this function as well.
\begin{align}
\tilde{H}=
\left( \begin{array}{ccc}
a & b & c \\
d & e & f \\
g & h & j
\end{array} \right)
\end{align}

  \subsection{Deconditioning}
In order to get back to the original coordinates, a final transformation matrix $H$ 
that is Euclidean normalized is calculated as follows:
\begin{align}
H=T'^{-1}\tilde{H}T
\end{align}

In the code, the following function realizes the solution:
\begin{lstlisting}
decondition(tBase, tAttach, H);
\end{lstlisting}

After this number of sub steps, the two necessary homography matrices are computed and can be used
afterwards.

\section{Projective Rectification:}

The estimated homographies are used to adapt the first and the third image to the second one.
Because of the fact, that our pictures are projective distorted, they can be rectified by
applying our algorithm.

The result of the first process can be seen in fig. \ref{fig:../../../target/general_result_1}.

\image{../../../target/general_result_1}{0.9}{%
		Resulting picture 1 (\emph{general positions}).}


The final result of the second adaption processes can be seen in the section of Visualization.
%\begin{lstlisting}
%cvEqualizeHist ( img1, out );
%\end{lstlisting}
%\begin{align}
%f
%\end{align}

\section{Visualization:}

In our implementation, the produced panorama image is shown on the screen and saved to disk.
Based on the three input pictures, it can be seen in fig. \ref{fig:../../../target/general_result_2}.

\image{../../../target/general_result_2}{0.9}{%
		Resulting picture 2 (\emph{general positions}).}

\section{Optional:}

%Change code in order to process arbitrary number of images.
How could one find corresponding point pairs automatically?

A first approach could be to determine point pairs dependend on the intersection of edges.
On overlapping sections of certain images, there would a structure of edges be detectable,
that could be used for the automatic point pair search. 
For this, it has to be considered, that a sufficient amount of points is available to
recognize similar edge structures and to ensure that structures of non overlapping parts
don't fit at all.

Independently on the camera, one could project an infrared raster in the scene.
Hence, the same points could be detected automatically with help of this grid system.

\section{Printed Code:}

%\lstset{language=<C++>}
%\begin{lstlisting}
%test
%\end{lstlisting}
\lstinputlisting{../native/pcv2.cpp}
\end{document}

